\chapter{Data Analysis} \label{Chapter5}
Answering the sub-questions

\section{SQ1: What data types, available from the control system, can predict maintenance activities?} \label{SQ1}
All different types of relevant data over time are listed in section \ref{Data Gathering}. For every type of data it is determined if it is able to predict upcoming maintenance. 

Robot arms exhibit very complex dynamic behavior, and different defects can affect this behavior. Also, their motion is completely different from that of rotating machines (or other continuously moving machines), for which the majority of present condition monitoring systems have been designed. \citet{Jaber2017} describes that data subtracted from a robot arm are transitory and last for a very short time. This is in contradiction with the rotating machines discussed widely in literature that emit continuous signals during their operation. The challenge here is how to design a reliable and intelligent condition monitoring system to be able to deal with the transitory and non-transitory signals for accurate diagnosis and prognosis.Therefore, the signals captured and features extracted have to be analyzed and classified in an appropriate way to provide an unambiguous identification of a faulty robot part before a catastrophic failure occurs.

\section{SQ2: How can the output data be interpret in order to predict maintenance?} \label{SQ2}
In the previous subsection, it was found that .... data is able to predict upcoming maintenance. In this subsection, it is determined how this data should be interpret in order to build a model. Firstly, a brief literature study discusses various CBM modeling techniques in order to determine which one is most suitable for this project.

\subsection{Literature on CBM techniques} \label{CBM techniques}
Currently, the OTD of Philips uses direct inspection to measure the condition of robot arms and to find defective faults. By applying CBM, these direct inspections are replaced by condition monitoring using sensors within the robot arm. In Section \ref{SQ1}, it is determined what data types should be analyzed to achieve a reliable maintenance prediction. 

Electric motor drives (sometimes called inverters or amplifiers) use power transistors to deliver energy. These transistors produce voltage; current is generated by the electrical circuit that is formed from the drive voltage and the motor windings. Because the power transistors produce voltage, a current loop is required to achieve precise control of current. A current loop compares the current command to the feedback and adjusts the drive voltage to minimize the error. Not all drives use current loops. Many drives output a voltage and rely on motor parameters to limit the current. This works in some applications, but usually not with higher power (> 500 W) or on more demanding machines.

\section{SQ3: On what performance indicators should the artifact be evaluated?} \label{SQ3}
The model should be checked on its correctness. If it is hard to determine this, estimations have to be made.
\section{SQ4: What are requirements for software engineers to integrate the artifact in the control system?} \label{SQ4}
The validated model should be implemented into the RB34 and later in other RACs and therefore software requirements should be determined. Maybe a conversation with Frank Velthuis, a software engineer from Beenen, will bring some new insights forward.
